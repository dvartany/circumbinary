\documentclass{article}
\usepackage[hang,bf,small]{caption}
\usepackage{hyperref}
\usepackage[all]{hypcap}

\setlength{\topmargin}{-.5in}
\setlength{\textheight}{9in}
\setlength{\oddsidemargin}{.125in}
\setlength{\textwidth}{6.25in}
\usepackage{graphicx}
\usepackage{caption}
\usepackage{amsmath}
\usepackage{subcaption}
\usepackage{amssymb}


\begin{document}
\begin{flushleft}
Jose Garmilla \\
David Vartanyan\\
Thermodynamics of Circumbinary Disks\\
\today\\
\end{flushleft}
\section{Introduction}
We consider the evolution of a circumbinary disks. We do not yet allow for a dead zone in the disk.

The thermal timescale is much smaller than the viscous timescale in a circumbinary disk, and so we approximate the temperature in equilibrium.

\section{Thermodynamics}
We allow for heating of the disk due to viscous dissipation, irradiation from the binary, and dissipation of density waves due to a tidal torque.

Assuming thermal equilibrium, we have, at low optical depths, $\tau \ll 1$,

\begin{equation}
\tau \sigma T_m^4 = F_{tid} + F_{\nu} + \tau F_{irr}\,,
\end{equation}

and at high optical depths, $\tau \gg 1$,

\begin{equation}
\sigma T_m^4 = F_{tid} + \frac{3\tau}{4}  F_\nu + F_{irr}\,.
\end{equation}

Extrapolated to all optical depths,

\begin{equation} \label{eq:encon}
\sigma T_m^4= \left(\frac{3\tau}{4} + \frac{1}{\tau}\right) \left(F_{v} + F_{tid}\right) + F_{irr}\,,
\end{equation}

where $T_m$ the disk midplane temperature. The optical depth is defined as $\tau =\frac{1}{2} \kappa \Sigma$, where $\kappa$ is the opacity.

The tidal heating term per surface is

\begin{equation}
F_{tid} = \frac{1}{2}\left(\Omega_b - \Omega\right) \Lambda \Sigma\,,
\end{equation}

where $\Omega_b = \sqrt{G M/a^3}$ is the binary angular velocity, with $a$ the binary semimajor axis, $\Omega = \sqrt{G M/r^{3}}$ is the Keplerian velocity, $M$ is the total mass of the binary, and $\Lambda$ is the torque due to the binary. At radius $r > h(r) + a$,

\begin{equation}
\Lambda = f \frac{q^2 GM}{a} \left(\frac{a}{r-a}\right)^4\,,
\end{equation}

where $q$ is the binary mass fraction and $f$ is a correction factor that fixes the inner disk truncation radius. Note that Martin 2013 fix $f =1$, which places their disk inner edge further out. 

The disk height $h(r)$ is given by

\begin{equation}
\frac{h}{r} = \left(\frac{k\,T\,r}{G M\mu}\right)^\frac{1}{2} \,,
\end{equation}

by assuming hydrostatic equilibrium for a gas-pressure dominated disk.
The heating due to irradiation from the binary is

\begin{equation}
F_{irr} = \frac{1}{2} \frac{L_c}{4\pi r^2} \zeta\,,
\end{equation}

where $\zeta$ is the incidence angle of radiation from the binary onto the disk at a height $\eta h$. Far from the binary, we can approximate $\zeta$ as

\begin{equation}
\zeta = r \frac{d}{dr} \left(\frac{\eta h}{r}\right)\,.
\end{equation}

The heating due to viscous dissipation is

\begin{equation}
F_\nu = \frac{3}{8\pi} \frac{F_J \Omega}{r^2} = \frac{9}{8} \Omega^2 \nu \Sigma\,,
\end{equation}

where the second equality comes from the definition of the angular momentum flux $F_J = 3 \pi \nu \Sigma l$, with $l = \Omega r^2$, the specific angular momentum. The viscosity is expressed as

\begin{equation}
\nu = \alpha \frac{c_s^2}{\Omega}
\end{equation}

where the sound speed is $c_s = \sqrt{\frac{P}{\rho}}$ and the viscosity is for a gas-pressure dominated disk 
 
\begin{equation}
v = 	\alpha \frac{k_B T}{\mu \Omega}\,.
\end{equation}

\section{Temperature Interpolation}

We can simplify the energy balance equation, Eq.\,\ref{eq:encon} by solving for the temperature in the various heating processes and then interpolating our results as 

\begin{equation}
T_m^4 = \left(T_{tid, thin}^ 4 + T_{tid, thick}^4\right) + \left(T_{\nu, thin}^4 + T_{\nu, thick}^4\right) + T_{irr}^4\,,
\end{equation}

where, for example, $F_{irr} = \sigma T_{irr}^4$. 

In the limit of small optical depth, $\tau \ll 1$, and heating dominated by viscous dissipation at the disk midplane, 

\begin{equation}
T_{\nu, thin} = \left(\frac{9}{4} \frac{\alpha\,k_b}{\sigma\,\mu\,\kappa_o} \Omega\right)^\frac{1}{3+\beta}\,.
\end{equation}

In the limit if small optical depth but heating dominated by tidal heating, also at the disk midplane,

\begin{equation}
T_{tid,thin} = \left((\Omega_b - \Omega)\frac{\Lambda}{\kappa_o \sigma} \right)^\frac{1}{4+\beta}\,.
\end{equation}

In the limit of large optical depth, $\tau \gg 1$, and heating dominated by viscous dissipation,

\begin{equation}
T_{\nu, thick} = \left(\frac{27}{64} \frac{\kappa_o\,\alpha\,k_b}{\sigma\,\mu} \Omega \Sigma^2\right)^\frac{1}{3-\beta}\,.
\end{equation}

In the limit of large optical depth and heating dominated by tidal dissipation,

\begin{equation}
T_{tid, thick} = \left(\frac{ 3 \kappa_o}{16}(\Omega_b - \Omega)\frac{\Lambda \Sigma^2}{2}\right)^\frac{1}{4-\beta}\,.
\end{equation}

The irradiation temperature is independent of opacity,
 
\begin{equation}
T_{irr} = \left( \left(\frac{\eta}{7} \frac{L_c}{4 \pi \sigma}\right)^2 \frac{k_b}{\mu G M}\right)^\frac{1}{7} r^{-\frac{3}{7}}
\end{equation}

\section{Opacity}

Following the prescription in Bell \& Lin 2004, we will assume a power-law for the opacity,

\begin{equation}
\kappa = \kappa_o T^\beta\,.
\end{equation} 

We use a piecewise continuous opacity  for different temperatures,

$$\kappa =
\begin{cases}
2\times 10^{-4} T^2, & \textrm{T} \leq 166.81\, \textrm{K} \\
2\times 10^{16} T^{-7}, & 166.81\, \textrm{K} <  \textrm{T} < 202.677\, \textrm{K} \\
0.1\times T^{1/2}, & \textrm{T} \geq 202.677\, \textrm{K}
\end{cases}
$$

The last expression for opacity holds up to greater than $2000$\,K, higher than disk temperatures we will consider.

\section{Disk Evolution}

We can now evaluate the disk evolution equation by expressing viscosity as a function of surface density $\Sigma$ and radius $r$.

The disk evolution equation is

\begin{equation} \label{eq:diskev}
\frac{\partial \Sigma}{\partial t} = \frac{1}{r} \frac{\partial}{\partial r} \left[3 r^{1/2} \frac{\partial}{\partial r} \left(r^{1/2} \nu \Sigma\right) - \frac{2 \Lambda \Sigma r^{3/2}}{(G M)^{1/2}}\right]\,.
\end{equation}

\section{Conservation Equations}

The equation for mass conservation is

\begin{equation}
M = \int 2\pi r \Sigma\, dr\,,
\end{equation}

and for momentum conservation,

\begin{equation}
\textit{L} = \int2\pi r \Sigma\, r^2 \Omega\, dr\,.
\end{equation}

We express the evolution equation in a conservative form, in terms of $\Sigma$ rather than $F_J$. This will facilitate verifying conservation every time step of the numerical procedure.

\section{Initial and Boundary Conditions}

We will set $\Sigma = 0$ for our inner boundary condition. Our outer boundary condition is irrelevant because our solution will never reach it, so we can likewise set $\Sigma = 0$ at our outer boundary condition. 

We can derive a more general inner boundary condition with the hypothesis $\partial \dot{M}/ \partial l = 0$, or equivalently,

\begin{equation}
\frac{\partial}{\partial l} \left(3 r^{1/2} \frac{\partial}{\partial r} \left(r^{1/2} \nu \Sigma\right) - \frac{2 \Lambda \Sigma r^{3/2}}{(G M)^{1/2}}\right)= 0\,.
\end{equation}

The binary torque $\Lambda$ vanishes to 0 at the inner edge. If we further assume that the accretion rate $\dot{M}$ is 0, we get that $\Sigma$ must be a constant, which we can fix to 0 and return our inner boundary condition above.

\section{Steady State Solution}

From Eq.\,\ref{eq:diskev}, we readily identify the accretion rate,

\begin{equation}
\frac{\dot{M}}{2\pi} = 3 r^{1/2} \frac{\partial}{\partial r} \left(r^{1/2} \nu \Sigma\right)\,,
\end{equation}

in the limit where the external torque $\Lambda \rightarrow 0$. To find the steady-state solution, we set the LHS of Eq.\,\ref{eq:diskev} to 0 and integrate to find the surface density,

\begin{equation}
\Sigma = -\frac{2}{3 \nu} \left(A+  \frac{B}{\sqrt{G M r}}\right)\,,
\end{equation}

where $A, B$, are integration constants. The integration constant $A$ satisfies the constraint $A = \frac{\dot{M}}{2\pi}$, where the accretion rate $\dot{M}$ is constant. For the standard constant accretion disk, the accretion rate satisfies $\dot{M} = 3\pi \nu \Sigma$, and this constrains the integration constant $B = 0$.

\section{Numerical Method}

We will use FiPy to solve Eq.\,\ref{eq:diskev}. FiPy is a finite volume package for Python, it's typically used to
solve PDEs that arise from conservation laws in physical sciences and engineering.

Since FiPy only works with conservation equations, we have to cast our evolution equation as a conservation law.
By inspecting equation \ref{eq:diskev} we can see that it can be written as a conservation law with a convection 
term as follows
\begin{equation} \label{eq:conservation}
\frac{\partial \Sigma}{\partial t} = -\frac{1}{r} \frac{\partial}{\partial r} \left( r \Sigma v_{r}\right)=-\nabla\cdot (\Sigma v_r \bold{e}_{r}),
\end{equation}
where
\begin{equation} \label{eq:vr}
v_r = -\frac{3}{r^{1/2}\Sigma} \frac{\partial}{\partial r} \left(r^{1/2} \nu \Sigma\right) + \frac{2 \Lambda r^{1/2}}{(G M)^{1/2}}.
\end{equation}
This way we have a conservation law. The conserved quantity is $\Sigma$, and the flux is $\Sigma v_r \bold{e}_r$.

The finite volume techinque consists on dividing the volume into small cells where the variable of interest ($\Sigma$) has a constant value,
and the value of the flux ($\Sigma v_r$) has a constant value at the cells' faces. The geometry of the cells doesn't have to be cartesian, in
our case we'll use cells shaped like annuli to take advantage of the cylindrical symmetry.
The discretization is obtained by integrating equation \ref{eq:conservation} over a volume cell and using the divergence theorem on the
right hand side to get
\begin{equation} \label{eq:discretization}
\frac{\Sigma_c(t+\Delta t) V_C - \Sigma_c(t) V_C}{\Delta t} =
-\sum_{f} \Sigma_{f}(t) v_{rf}(t) A_{f} (\bold{e}_r \cdot \bold{e}_{f}),
\end{equation}
where the subscript $c$ means that quantity is evaluated at the cell center and $f$ means the quantity is evaluated at one of the faces,
$V_C$ is the cell's volume, $A_f$ is the area of face $f$, and $\bold{e}_{f}$ is the normal vector of the face $f$ pointing outwards.
In our case we only have two faces per cell because the grid is one 
dimensional.
\end{document}
